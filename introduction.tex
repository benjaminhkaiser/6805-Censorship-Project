\section{Introduction}
The Bitcoin cryptocurrency relies on a shared, global transaction ledger that each participant stores a copy of: a blockchain. We set out to investigate whether Internet censorship, which is widely utilized by many countries, might interfere with the Bitcoin blockchain. In particular, we were curious if someone intent on destabilizing the blockchain could craft a Bitcoin block that would be filtered by a country's censorship apparatus and what the effects on the blockchain might be in such a scenario.

We focused our efforts on the censorship conducted by China, known as the Great Firewall of China (GFW). We were unable to construct a block that would be filtered in practice by the GFW. In order to pursue the idea further, we conducted a research survey and some experiments to determine how the GFW functions and configured a Bitcoin simulator to test the impact of a network partition caused by a block being filtered at a country's border. We believe our results are of interest because Internet censorship techniques are always changing, and it is conceivable that some techniques susceptible to such an attack may be deployed.

\colorbox{yellow}{We can add a brief overview of our results here once}
\colorbox{yellow}{we have them}
