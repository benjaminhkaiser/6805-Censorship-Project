\section{Introduction}
The Bitcoin cryptocurrency relies on a shared, global transaction ledger that each participant stores a copy of: a blockchain. We set out to investigate whether Internet censorship, which is widely utilized by many countries, might interfere with the Bitcoin blockchain by preventing those behind the wall of censorship from reaching consensus with those outside the wall. In particular, we were curious if someone intent on destabilizing the blockchain could craft a Bitcoin block that would be filtered by a country's censorship apparatus and what the effects on the blockchain might be in such a scenario.

We focused our efforts on the censorship conducted by China, known as the Great Firewall of China (GFW). We first conducted a research survey and performed some tests using a virtual machine located within China to determine how the GFW functions. Unfortunately, we were unable to construct a block that would be filtered in practice by the GFW. In order to pursue the idea further, we next conducted experiments to test the impact of a network partition caused by a block being filtered at a country's border. We believe our results are of interest because Internet censorship techniques are always changing, and although the GFW is not currently censoring Bitcoin traffic, it is conceivable that it or another censorship regime may at some point.
