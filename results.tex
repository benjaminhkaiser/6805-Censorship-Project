\subsection{Results}

\subsubsection{Firewall Testing}
To test the GFW we first had to obtain access to a system with a censored Internet connection. When a message caused the GFW to block a connection, the connection was reset and the CN machine was unable to connect to the US machine on the same port for a few minutes.

We attempted to construct a Bitcoin block that would be filtered by the GFW. Arbitrary data has been embedded into blocks before; the most common places that such data is stored in a block are the address fields (by encoding the data in hex), the coinbase transaction, or the scripts\footnote{In one notable case, a simple cross-site scripting (XSS) attack was embedded in a block and was apparently demonstrated to work on blockchain.info\cite{reddit}, although the site has since been patched to properly escape HTML.}. 

In the end we determined that we would need to trick the GFW into interpreting a block as Web traffic of some sort (e.g., DNS or HTTP), which was not feasible.

\subsubsection{Simulation}
Initial analysis of the Bitcoin Simulator built on NS3 showed that the simulator didn't simulate at a high enough level of fidelity for us to conduct our desired experiments. Instead, we focused on using Shadow.

We compiled version 0.9.2 of the \texttt{bitcoind} client\footnote{Shadow required us to use a slightly modified version of 0.9.2 available at https://github.com/amiller/bitcoin/tree/0.9.2-netmine} for use in Shadow. The published instructions for running Shadow's Bitcoin Plugin were significantly outdated, so we created a Docker container that builds their simulator and applies patches so that their original experiment can be recreated\footnote{The Dockerfile that builds this container is available at https://github.com/AndrewFasano/shadow-bitcoin-docker}. In our Docker container, we ran a network of 4 bitcoind clients communicating as we broadcast transactions and blocks to one of the clients.

We patched the \texttt{bitcoind} client to drop messages it receives containing a ``blocked'' keyword.

We reconfigured Shadow to run a network with two groups of two nodes that were unable to communicate with each other. We then sent blocks to only one of these groups, expecting the clients in that group to have a different blockchain than the clients in the other group.We were unable to configure Shadow to run in this fashion due to time limitations.

If we were to get Shadow to separately partition two network segments, we would have then modified how Shadow sends messages between clients to drop any messages between the two groups where the message contains our blocked keyword.
