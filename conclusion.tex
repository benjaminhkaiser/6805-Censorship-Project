\section{Discussion}
Although we do not believe Bitcoin is presently censored by nation states, we believe it could be a subject of scrutiny in the future. Given that the majority of hash power resides in one country, the conscious decision to censor Bitcoin activity could have detrimental effects to the network as a whole. What originated as a decentralized currency has manifested in a concentration of power in one geographic region. This leaves the network vulnerable to decisions made by regimes that can choose to block not just blacklisted keywords but also legitimate payments in a variety of methods based on the attack we described.

Thankfully, censoring every device within a geographic region is a big challenge. Multiple ways around censorship firewalls have been proven to work in the past; e.g., web proxies, VPNs, Tor, etc. [citation needed]. Bitcoin nodes nowadays even resort to using a variety of protocols to communicate block information, such as Bitcoin FIBRE~\cite{fibre} and Compact Blocks~\cite{compact-blocks}. Fully blocking communication between miners on two sides of a censorship wall is still a difficult feat as censored nodes would only need one point of communication that can bridge the two sides. Of course, this would be at the cost of added network latency, which has economic and network implications on its own. If this attack were to come to fruition, the organization of mining pools could help mitigate against its effects. If the pool manager were to relocate to another location that is not subject to censorship, the manager could still select blocks and relay them back to miners located within the firewall.

Given that Bitcoin is not currently subject to censorship, this creates an opportunity for individuals to add blacklisted messages onto the blockchain. Though there are already many tactics to circumvent the censorship of information, the blockchain could provide an additional outlet.
