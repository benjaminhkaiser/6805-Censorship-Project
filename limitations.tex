\section{Limitations}
Speaking broadly, experiments performed on a model are only useful to the degree to which the model accurately represents reality.
Therefore, we want to make explicit the limitations of our model and the simplifying assumptions that we made when designing our simulations. These limitations are as follows:

%We plan to conduct the experiments descrbed in Section~\ref{sec:design}
%Even though we were not able to instantiate our model or run our simulations, we plan to do so in the future, so
\begin{itemize}
	\item \textit{Mining pools}: In practice, mining pools in which miners cooperate to find new blocks and share the rewards are common \cite{Rosenfeld}. We model mining pools as a single miner who has the cumulative hash power of the pool. This fails to account for scenarios in which miners outside of the GFW contribute to a pool that is managed by a node inside of the GFW.
	\item \textit{Miner intervention}: If the blockchain were to fork due to a mined block that exploits a censorship system, Bitcoin users would certainly notice and possibly attempt to rectify the situation.
%If a cryptocurrency network experiences a large-scale fork that is not resolved quickly the currency could destabilize and lose value.
In attempt to stabilize the currency, it is possible that a large group of miners could quickly react to a fork and behave in an unexpected way. For instance, in a scenario where a majority of mining power is uncensored, an uncensored group of miners pool could modify their clients to ignore blocks that would be censored. If by doing this, they shift a majority of mining power to behave as if it were censored, the entire Bitcoin network will eventually reach agreement on a blockchain that does not include the malicious block.
	\item \textit{Non-Standard Clients}: While~\cite{shadow-bitcoin} shows that the Bitcoin reference client is used by a majority of Bitcoin users, it does not specify the percentage of mining power that is using the client. This attack would not have the same affect on miners using non-standard clients that do not require validation of blocks or use a different protocol or encoding to share blocks.
\end{itemize}


