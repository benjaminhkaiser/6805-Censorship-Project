\section{Experiment Design}

To test our hypothesis, we needed to conduct two experiements to understand how the the Great Firewall blocks traffic and how the Bitcoin client behaves when certian blocks cannot be propigated between two groups of clients.

\subsection{Firewall Testing}
To check if the GFW would prevent the propigation of a Bitcoin block from crossing through its censorship systems, we obtained two machines: one in China and one in the United States. To confirm that our CN machine was affected by the GFW, we attempted to access a blocked websites and confirmed they were inaccessable.

With our two machines, we attempted to send a variety of messages from our US machine to our CN machine and vice versa. We used the netcat UNIX utility to send ASCII messages. When we sent a message causing the GFW to block the connection, we expected the CN machine to be unable to reach the US machine on the same port for a few minutes, as described in~\ref{TODO}.

After validating that the GFW was affecting traffic between our two machines, we will send test messgaes of various formats containing known-blocked words. For each type of message, we will record if the connection is blocked.

\subsection{Simulation}
By simulating a small-scale version of the Bitcoin network with network censorship in place, we can examine how a censorship-induced fork would affect the network over time. The majority of Bitcoin clients run bitcoind, the reference client implementation of the Bitcoin protocol ~\cite{shadow-bitcoin}.

Data can be collected by simulating an entire bitcoin network with multiple nodes running bitcoind or by simulating the network traffic between nodes.

Shadow~\cite{shadow} running with it's Bitcoin plugin~\cite{shadow-bitcoin} emulates real bitcoind clients in a simulated network that can be used to conduct experiments.

Bitcoin Simulator~\cite{bitcoin-simulator} built on top of NS3~\cite{NS3} simulates network traffic between bitcoin clients. % TODO: more?

The possible scenarios we want to examine fall into four categories:
\begin{enumerate}
\item A majority of mining power is unaffected by censorship, a minority of mining power cannot send/recieve messages containing a blacklisted word.
\item A minority of mining power is unaffected by censorship, a majority of mining power cannot send/recieve messages containing a blacklisted word.
\item A majority of mining power is unaffected by censorship (Partition 1), a minority of mining power cannot send/recieve messages to/from Partition 1 containing a blacklisted word (Partition 2), a small number of miners is unaffected by censorship and can communicate to both Partitions 1 and 2 without any censorship.
\end{enumerate}
