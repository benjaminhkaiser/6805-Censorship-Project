\section{Experiments}
\colorbox{yellow}{Need some intro text here.}
we first tried to create a block that would get caught, then once we realized we couldn't do it in practice, we tried the NS3 bitcoin sim, which we determined abstracted away too many important details of the bitcoin client and network. So then we explored the bitcoin client's source code and moved to a more accurate simulator, Shadow.

\subsection{Triggering the GFW}
We attempted to construct a Bitcoin block that would be filtered by the GFW. Arbitrary data has been embedded into blocks before; the most common places that such data is stored in a block are the address fields (by encoding the data in hex), the coinbase transaction, or the scripts\footnote{In one notable case, a simple cross-site scripting (XSS) attack was embedded in a block and was apparently demonstrated to work on blockchain.info\cite{reddit}, although the site has since been patched to properly escape HTML.}. 

In the end we determined that we would need to trick the GFW into interpreting a block as Web traffic of some sort (e.g., DNS or HTTP), which was not feasible. \colorbox{yellow}{(To all, but especially Andrew: is there anything else we}
\colorbox{yellow}{should say here about what we tried / observed?)}

\subsection{NS3 Bitcoin Simulator}
\subsection{Bitcoin Reference Client Source}
\subsection{Shadow Simulator}